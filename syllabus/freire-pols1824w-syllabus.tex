\documentclass[12pt,]{article}
\usepackage[margin=1in]{geometry}
\newcommand*{\authorfont}{\fontfamily{phv}\selectfont}
\usepackage[]{libertine}
\usepackage{abstract}
\renewcommand{\abstractname}{}    % clear the title
\renewcommand{\absnamepos}{empty} % originally center
\newcommand{\blankline}{\quad\pagebreak[2]}
\usepackage[UKenglish]{babel}

\providecommand{\tightlist}{%
  \setlength{\itemsep}{0pt}\setlength{\parskip}{0pt}} 
\usepackage{longtable,booktabs}

\usepackage{parskip}
\usepackage{titlesec}
\titlespacing\section{0pt}{12pt plus 4pt minus 2pt}{6pt plus 2pt minus 2pt}
\titlespacing\subsection{0pt}{12pt plus 4pt minus 2pt}{6pt plus 2pt minus 2pt}

\titleformat*{\subsubsection}{\normalsize\itshape}

\usepackage{titling}
\setlength{\droptitle}{-.25cm}

%\setlength{\parindent}{0pt}
%\setlength{\parskip}{6pt plus 2pt minus 1pt}
%\setlength{\emergencystretch}{3em}  % prevent overfull lines 

\usepackage[T1]{fontenc}
\usepackage[utf8]{inputenc}

\usepackage{fancyhdr}
\pagestyle{fancy}
\usepackage{lastpage}
\renewcommand{\headrulewidth}{0.3pt}
\renewcommand{\footrulewidth}{0.0pt} 
\lhead{}
\chead{}
\rhead{\footnotesize Political Violence -- Fall 2019}
\lfoot{}
\cfoot{\small \thepage/\pageref*{LastPage}}
\rfoot{}

\fancypagestyle{firststyle}
{
\renewcommand{\headrulewidth}{0pt}%
   \fancyhf{}
   \fancyfoot[C]{\small \thepage/\pageref*{LastPage}}
}

%\def\labelitemi{--}
%\usepackage{enumitem}
%\setitemize[0]{leftmargin=25pt}
%\setenumerate[0]{leftmargin=25pt}

\usepackage{xcolor}
\definecolor{darkblue}{rgb}{0.0,0.0,0.55}
\exhyphenpenalty=1000
\hyphenpenalty=1000
\widowpenalty=1000
\clubpenalty=1000


\makeatletter
\@ifpackageloaded{hyperref}{}{%
\ifxetex
  \usepackage[setpagesize=false, % page size defined by xetex
              unicode=false, % unicode breaks when used with xetex
              xetex]{hyperref}
\else
  \usepackage[unicode=true]{hyperref}
\fi
}
\@ifpackageloaded{color}{
    \PassOptionsToPackage{usenames,dvipsnames}{color}
}{%
    \usepackage[usenames,dvipsnames]{color}
}
\makeatother
\hypersetup{breaklinks=true,
            bookmarks=true,
            pdfauthor={ ()},
             pdfkeywords = {},  
            pdftitle={Political Violence},
            colorlinks=true,
            citecolor=darkblue,
            urlcolor=darkblue,
            linkcolor=darkblue,
            pdfborder={0 0 0}}
\urlstyle{same}  % don't use monospace font for urls


\setcounter{secnumdepth}{0}





\usepackage{setspace}

\title{Political Violence}
\author{Danilo Freire}
\date{Fall 2019}


\begin{document}  

		\maketitle
		
	
		\thispagestyle{firststyle}

%	\thispagestyle{empty}


	\noindent \begin{tabular*}{\textwidth}{ @{\extracolsep{\fill}} lr @{\extracolsep{\fill}}}


E-mail: \href{mailto:danilofreire@brown.edu}{\nolinkurl{danilofreire@brown.edu}} & Web: \href{http://danilofreire.github.io/pols1824w}{danilofreire.github.io/pols1824w}\\
Office Hours: Mo-Fr Afternoon  &  Class Hours: Tuesday, 4-6:30pm\\
Office: 8 Fones Alley, 114  & Classroom: 101 Thayer Street (VGQ 1st fl) 116B\\
	&  \\
	\hline
	\end{tabular*}
	
\vspace{2mm}
	


\hypertarget{course-description}{%
\section{Course Description}\label{course-description}}

This course explores the main debates on the causes and consequences of
political violence. We will focus on three major topics: civil wars,
state-sponsored violence, and terrorism. Since the end of World War II,
domestic conflict has largely outpaced international wars as the
dominant type of violence. But what makes civil wars so prevalent in
recent years? What are the conditions under which a state decides to
attack its own citizens? Why some groups resort to terrorism while
others prefer nonviolent tactics?

The class has three goals. First, students will become familiar with the
literature on political violence, its most important debates and recent
findings. Second, students should be able to evaluate research methods
and critically assess distinct theoretical approaches. Lastly, the
course will develop the students' writing skills by asking them to
review academic articles and present a paper of their own by the end of
the semester.

\hypertarget{course-information}{%
\section{Course Information}\label{course-information}}

We will meet every Tuesday from 16:00 to 18:30 at
\href{http://brown.edu/Facilities/Facilities_Management/maps/index.php\#building/VGQUADA}{101
Thayer Street (VGQ 1st fl) 116B}. It is very important that you read the
assigned readings before class. Students are encouraged to engage in
critical discussions and are most welcome to express their views openly
and freely. I would suggest you to bring some notes to the class so that
we can discuss together the topics you find most interesting. Some of
the texts make use of statistical models and game theory, but don't be
intimidated by them. Feel free to skip the technical parts if they are
too challenging and focus on the main ideas of the readings.

All information about the course will be available at
\url{http://danilofreire.github.io/pols1824w}. The syllabus will be
updated periodically according to the progress of the class. Please
remember to visit the website regularly.

\hypertarget{office-hours}{%
\section{Office Hours}\label{office-hours}}

I am very flexible when it comes to office hours, but it is easier to
contact me via email. Feel free to send me a message any time at
\href{mailto:danilofreire@brown.edu}{\nolinkurl{danilofreire@brown.edu}}.
I will probably reply in a few hours. You can also meet me in the
afternoon at my office. I am in the Political Theory Project every
weekday and you can find me at \href{https://goo.gl/maps/8WhhyCNDHnw}{8
Fones Alley, first floor, office 114}. If possible, please send me an
email before coming to my office just to make sure two students will not
book the same time slot.

\hypertarget{community-standards}{%
\section{Community Standards}\label{community-standards}}

Brown University is committed to full inclusion of all students. Please
inform me early in the term if you have a disability or other conditions
that might require accommodations or modification of any of these course
procedures. You may speak with me after class or during office hours.
For more information, please contact
\href{https://www.brown.edu/campus-life/support/accessibility-services}{Student
and Employee Accessibility Services} at 401-863-9588 or
\href{mailto:SEAS@brown.edu}{\nolinkurl{SEAS@brown.edu}}. Students in
need of short-term academic advice or support can contact one of the
deans in the Dean of the College office.

\hypertarget{english-language-learners}{%
\section{English Language Learners}\label{english-language-learners}}

Brown University welcomes students from around the country and the
world, and the unique perspectives international and multilingual
students bring enrich the campus community. To empower multilingual
learners, an array of support is available including language and
culture workshops and individual appointments. For more information
about English Language Learning support at Brown, contact the ELL
Specialists at \url{ellwriting@brown.edu}. No student will be penalised
for their command of the English language.

\hypertarget{academic-integrity}{%
\section{Academic Integrity}\label{academic-integrity}}

Students will write three review reports and a longer essay for this
course. All writing should be your own work, and I take plagiarism very
seriously. I am happy to provide any help you may require with your
lessons as long as you are committed to the course. It is also important
to cite other people's work whenever necessary, and if in doubt, mention
your sources.

\hypertarget{special-needs}{%
\section{Special Needs}\label{special-needs}}

If you have any special needs, please contact me. I'm happy to make
necessary arrangements so you can follow this course.

\hypertarget{requirements-and-grading}{%
\section{Requirements and Grading}\label{requirements-and-grading}}

\textbf{Participation: 15\%.} Students should be active participants in
the course. Feel free to ask any question you may have, help others if
you know how, and make suggestions or comments you believe are
interesting. I hope we create a friendly, open environment for learning
and students are the most important part of it.

\textbf{Three Rewiew Reports: 45\%.} The reviews should be 3-5 pages
long. Imagine you are a reviewer for a good academic journal and think
of how you could help the author improve the article. Are the arguments
well-developed in the text? Is the research design plausible? What
further examples could the author include to strenghten his/her
arguments? Try to provide as much constructive criticism as possible.
You don't need to summarise the paper, just critically engage with it.
You should write one paper for each section of the course (civil wars,
mass killings, and terrorism), but you're free to choose any reading you
prefer. The essays are due at the beginning of the class and late
assignments will not be eligible for an A. Each report will account for
15\% of your grade.

\textbf{Final Project: 40\%.} In the final project, students will have
the opportunity to write a longer essay about a topic that interests
them. The essay should be related to the readings of the course, but you
are particularly encouraged to explore new ideas and use new data to
test their hypotheses. Students can work in groups of up to three people
as most academic research is currently done collaboratively. By the
second week of the course, students should submit a one-page summary of
their future essay. The instructor and two colleagues will review the
paper proposal and give the authors constructive feedback. Students will
then write a full draft during the term. In the last week of the course,
students will present their findings to the class and receive feedback
from their colleagues. The final paper is due one week after that.

\hypertarget{materials}{%
\section{Materials}\label{materials}}

We will not use a textbook in this course. Most readings are recent
academic articles published in political science journals. You probably
have access to the publications through Brown, but let me know if you
cannot find any of them. I will also include links to the book chapters
mentioned in the syllabus.

\hypertarget{schedule-and-readings}{%
\section{Schedule and Readings}\label{schedule-and-readings}}

All readings are available on
\href{https://github.com/danilofreire/political-violence-syllabus-2019}{the
course's GitHub repository}. It is very important that you read the
assigned readings before class. Students are encouraged to engage in
critical discussions and are most welcome to express their views openly
and freely. I would suggest you to bring some notes to the class so that
we can discuss together the topics you find most interesting. Some of
the texts make use of statistical models and game theory, but don't be
intimidated by them. Feel free to skip the technical parts if they're
too challenging and focus on the main ideas of the readings.

\hypertarget{week-1-introduction-and-course-overview.-long-term-trends-in-armed-conflict}{%
\subsection{Week 1: Introduction and course overview. Long-term trends
in armed
conflict}\label{week-1-introduction-and-course-overview.-long-term-trends-in-armed-conflict}}

\begin{itemize}
\tightlist
\item
  Rosner, M. 2019. \href{https://ourworldindata.org/war-and-peace}{Our
  World in Data: War and Peace}.
  \href{https://ourworldindata.org/terrorism}{Our World in Data:
  Terrorism}. \href{https://ourworldindata.org/genocides}{Our World in
  Data: Genocides}.
\item
  Pinker, S. 2013.
  \href{http://sci-hub.tw/https://academic.oup.com/isr/article-abstract/15/3/396/1851344?redirectedFrom=fulltext}{The
  Decline of War---The Main Issues}. \emph{International Studies
  Review}, 15(3):397-399.
\item
  Cirillo, P. \& Taleb, N. 2015.
  \href{https://www.fooledbyrandomness.com/pinker.pdf}{The Decline of
  Violent Conflicts: What Do the Data Really Say?} Nobel Foundation
  Symposium 161: The Causes of Peace.
\item
  Gohdes, A. \& Price, M. 2012.
  \href{http://journals.sagepub.com/doi/10.1177/0022002712459708}{First
  Things First: Assessing Data Quality before Model Quality}.
  \emph{Journal of Conflict Resolution}, 57(6):1090-1108.
\item
  Lacina, B. \& Gleditsch, N. 2012.
  \href{http://journals.sagepub.com/doi/abs/10.1177/0022002712459709}{The
  Waning of War is Real: A Response to Gohdes and Price}. \emph{Journal
  of Conflict Resolution}, 57(6):1109-1127.
\end{itemize}

\hypertarget{part-i-civil-wars}{%
\section{Part I: Civil Wars}\label{part-i-civil-wars}}

\hypertarget{week-2-conceptual-definitions-overview-of-the-recent-literature}{%
\subsection{Week 2: Conceptual definitions, overview of the recent
literature}\label{week-2-conceptual-definitions-overview-of-the-recent-literature}}

\begin{itemize}
\tightlist
\item
  Kalyvas, S. 2006. \emph{The Logic of Violence in Civil War}. New York:
  Cambridge University Press.
  \href{https://github.com/danilofreire/pols1824w/blob/master/readings/kalyvas2006chap1-2.pdf}{Chapters
  1 and 2}.
\item
  Sambanis, N. 2004.
  \href{http://journals.sagepub.com/doi/abs/10.1177/0022002704269355}{What
  Is Civil War? Conceptual and Empirical Complexities of an Operational
  Definition}. \emph{Journal of Conflict Resolution}, 48(6):814-858.
\item
  Walter, B. 2017.
  \href{https://www.annualreviews.org/doi/abs/10.1146/annurev-polisci-060415-093921}{The
  New New Civil Wars}. \emph{The Annual Review of Political Science},
  20:469-486.
\item
  Cederman, L-E. \& Vogt, M. 2017.
  \href{http://journals.sagepub.com/doi/abs/10.1177/0022002717721385}{Dynamics
  and Logics of Civil War}. \emph{Journal of Conflict Resolution},
  61(9):1-25.
\end{itemize}

\hypertarget{week-3-causes-of-civil-war}{%
\subsection{Week 3: Causes of civil
war}\label{week-3-causes-of-civil-war}}

\begin{itemize}
\tightlist
\item
  Collier, P. \& Hoeffler, A. 2004.
  \href{https://academic.oup.com/oep/article-abstract/56/4/563/2361902}{Greed
  and Grievance in Civil War}. \emph{Oxford Economic Papers},
  56(4):563-595.
\item
  Fearon, J. \& Laitin, D. 2003.
  \href{https://www.cambridge.org/core/journals/american-political-science-review/article/ethnicity-insurgency-and-civil-war/B1D5D0E7C782483C5D7E102A61AD6605}{Ethnicity,
  Insurgency, and Civil War}. \emph{American Political Science Review},
  97(1):75-90.
\item
  Wimmer, A., Cederman, L-E. \& Min, B. 2009.
  \href{https://doi.org/10.1177\%2F000312240907400208}{Ethnic Politics
  and Armed Conflict: A Configurational Analysis of a New Global Data
  Set}. \emph{American Sociological Review}, 74(2):316-337.
\item
  Kalyvas, S. \& Balcells, L. 2010.
  \href{https://doi.org/10.1017/S0003055410000286}{International System
  and Technologies of Rebellion: How the End of the Cold War Shaped
  Internal Conflict}. \emph{American Political Science Review},
  104(3):415-429.
\item
  Ward, M., Greenhill, B., \& Bakke, K. 2010.
  \href{http://journals.sagepub.com/doi/abs/10.1177/0022343309356491}{The
  Perils of Policy by P-Value: Predicting Civil Conflicts}.
  \emph{Journal of Peace Research}, 47(4):363-375.
\end{itemize}

\hypertarget{week-4-violence-against-civilians}{%
\subsection{Week 4: Violence against
civilians}\label{week-4-violence-against-civilians}}

\begin{itemize}
\tightlist
\item
  Kalyvas, S. 1999.
  \href{http://journals.sagepub.com/doi/abs/10.1177/104346399011003001}{Wanton
  And Senseless? The Logic of Massacres in Algeria}. \emph{Rationality
  and Society}, 11(3):243-285.
\item
  Lacina, Bethany. 2006.
  \href{http://journals.sagepub.com/doi/abs/10.1177/0022002705284828}{Explaining
  the Severity of Civil Wars}. \emph{Journal of Conflict Resolution},
  50(2):276-289.
\item
  Humphreys, M. \& Weinstein, J. 2006.
  \href{https://www.cambridge.org/core/journals/american-political-science-review/article/handling-and-manhandling-civilians-in-civil-war/4981647EDFE411635EB2086C4A851BFA}{Handling
  and Manhandling Civilians in Civil War}. \emph{American Political
  Science Review}, 100(3):429-447.
\item
  Cohen, D. 2013.
  \href{https://www.cambridge.org/core/journals/american-political-science-review/article/explaining-rape-during-civil-war-crossnational-evidence-19802009/30FC323D6DA7E923547156CC0E947213}{Explaining
  Rape During Civil War: Cross-National Evidence (1980-2009)}.
  \emph{American Political Science Review}, 107(3):461-477.
\item
  Balcells, L. 2010.
  \href{https://academic.oup.com/isq/article-abstract/54/2/291/1793034}{Rivalry
  and Revenge: Violence against Civilians in Conventional Civil Wars}.
  \emph{International Studies Quarterly}, 54(2):291-313.
\end{itemize}

\hypertarget{week-5-ending-civil-wars}{%
\subsection{Week 5: Ending civil wars}\label{week-5-ending-civil-wars}}

\begin{itemize}
\tightlist
\item
  Walter, B. 1997.
  \href{https://www.cambridge.org/core/journals/international-organization/article/critical-barrier-to-civil-war-settlement/AF2E36B866EC5E658266D01C5B00B42F}{The
  Critical Barrier to Civil War Settlement}. \emph{International
  Organization}, 51(3):335-364.
\item
  Howard, L. \& Stark, A. 2018.
  \href{https://www.mitpressjournals.org/toc/isec/42/3}{How Civil Wars
  End: The International System, Norms, and the Role of External
  Actors}. \emph{International Security}, 42(3):127--171.
\item
  Toft, M. 2010.
  \href{https://www.mitpressjournals.org/doi/abs/10.1162/isec.2010.34.4.7}{Ending
  Civil Wars: A Case for Rebel Victory?}. \emph{International Security},
  34(4):7-36.
\item
  Fortna, V. 2004.
  \href{https://academic.oup.com/isq/article/48/2/269/1888752}{Does
  Peacekeeping Keep Peace? International Intervention and the Duration
  of Peace After Civil War}. \emph{International Studies Quarterly},
  48(2):269-292.
\item
  Findley. M. \& Young, J. 2015.
  \href{https://www.journals.uchicago.edu/doi/abs/10.1086/682400}{Terrorism,
  Spoiling, and the Resolution of Civil Wars}. \emph{Journal of
  Politics}, 77(4):1115-1128.
\end{itemize}

\hypertarget{part-ii-state-sponsored-violence}{%
\section{Part II: State-Sponsored
Violence}\label{part-ii-state-sponsored-violence}}

\hypertarget{week-6-what-are-genocides-and-politicides}{%
\subsection{Week 6: What are genocides and
politicides?}\label{week-6-what-are-genocides-and-politicides}}

\begin{itemize}
\tightlist
\item
  United Nations. 2019.
  \href{https://www.un.org/en/genocideprevention/genocide.shtml}{Office
  on Genocide Prevention and the Responsibility to Protect:
  Definitions}.
\item
  Huttenbach, H. 1988.
  \href{https://academic.oup.com/hgs/article/3/3/289/602465}{Locating
  the Holocaust on the Genocide Spectrum: Towards a Methodology of
  Definition and Categorization}. \emph{Holocaust and Genocide Studies},
  3(3):289-303.
\item
  Fein, H. 1993.
  \href{http://booksandjournals.brillonline.com/content/journals/10.1163/157181193x00013}{Accounting
  for Genocide after 1945: Theories and Some Findings}.
  \emph{International Journal on Minority and Group Rights},
  1(2):79-106.
\item
  Levene, M. 2000. \href{https://muse.jhu.edu/article/18358/pdf}{Why Is
  the Twentieth Century the Century of Genocide?}. \emph{Journal of
  World History}, 11(2):305-336.
\item
  Blatman, D. 2015.
  \href{https://www.tandfonline.com/doi/abs/10.1080/14623528.2015.991206}{Holocaust
  Scholarship: Towards a Post-Uniqueness Era}. \emph{Journal of Genocide
  Research}, 17(1):21-43.
\end{itemize}

\hypertarget{week-7-no-classes}{%
\subsection{Week 7: No classes}\label{week-7-no-classes}}

\hypertarget{week-8-cross-national-determinants-of-genocide}{%
\subsection{Week 8: Cross-national determinants of
genocide}\label{week-8-cross-national-determinants-of-genocide}}

\begin{itemize}
\tightlist
\item
  Harff, B. 2003.
  \href{https://www.cambridge.org/core/journals/american-political-science-review/article/no-lessons-learned-from-the-holocaust-assessing-risks-of-genocide-and-political-mass-murder-since-1955/FBA37FA563AC35E1CB6F7B57F8140F2C}{No
  Lessons Learned from the Holocaust? Assessing Risks of Genocide and
  Political Mass Murder since 1955}. \emph{American Political Science
  Review}, 97(1):57-73.
\item
  Uzonyi, G. 2014.
  \href{http://journals.sagepub.com/doi/abs/10.1111/1467-9248.12181}{Domestic
  Unrest, Genocide and Politicide}. \emph{Political Studies},
  64(2):1-20.
\item
  Valentino, B., Huth, P. \& Balch-Lindsay, D. 2004.
  \href{https://www.cambridge.org/core/journals/international-organization/article/draining-the-sea-mass-killing-and-guerrilla-warfare/A4DD186DD876B363754AD358B8148014}{``Draining
  the Sea'': Mass Killing and Guerrilla Warfare}. \emph{International
  Organization}, 58(2):375-407.
\item
  Ahram, A. 2014.
  \href{https://www.tandfonline.com/doi/abs/10.1080/09546553.2012.734875}{The
  Role of State-Sponsored Militias in Genocide}. \emph{Terrorism and
  Political Violence}, 26(3):488-503.
\item
  Ulfelder, J. 2013.
  \href{https://papers.ssrn.com/sol3/papers.cfm?abstract_id=2303048}{A
  Multimodel Ensemble for Forecasting Onsets of State-Sponsored Mass
  Killing}.
\end{itemize}

\hypertarget{week-9-preventing-genocides}{%
\subsection{Week 9: Preventing
genocides}\label{week-9-preventing-genocides}}

\begin{itemize}
\tightlist
\item
  Bellamy, A. 2015.
  \href{http://journals.sagepub.com/doi/abs/10.1177/0022343315569333}{When
  States Go Bad: The Termination of State Perpetrated Mass Killing}.
  \emph{Journal of Peace Research}, 52(5):565-576.
\item
  De Waal, A., Meierhenrich, J. \& Conley-Zilkic, B. 2012.
  \href{http://heinonline.org/HOL/LandingPage?handle=hein.journals/forwa36\&div=7\&id=\&page=}{How
  Mass Atrocities End: An Evidence-Based Counter-Narrative}. \emph{The
  Fletcher Forum of World Affairs}, 36(1):15-31.
\item
  Melander, E. 2009.
  \href{http://journals.sagepub.com/doi/abs/10.1177/0738894209106482}{Selected
  To Go Where Murderers Lurk? The Preventive Effect of Peacekeeping on
  Mass Killings of Civilians}. \emph{Conflict Management and Peace
  Science}, 26(4):389-406.
\item
  Krain, M. 2005.
  \href{https://academic.oup.com/isq/article/49/3/363/1934221}{International
  Intervention and the Severity of Genocides and Politicides}.
  \emph{International Studies Quarterly}, 49(3):363-387.
\item
  Krain, M. 2017.
  \href{https://www.tandfonline.com/doi/abs/10.1080/14623528.2016.1240516}{The
  Effect of Economic Sanctions on the Severity of Genocides or
  Politicides}. \emph{Journal of Genocide Research}, 19(1):88-111.
\end{itemize}

\hypertarget{part-iii-terrorism}{%
\section{Part III: Terrorism}\label{part-iii-terrorism}}

\hypertarget{week-10-concepts-again-what-is-terrorism}{%
\subsection{Week 10: Concepts, again: what is
terrorism?}\label{week-10-concepts-again-what-is-terrorism}}

\begin{itemize}
\tightlist
\item
  Weinberg, L., Pedahzur, A. \& Hirsch-Hoeffler, S. 2004.
  \href{https://www.tandfonline.com/doi/pdf/10.1080/095465590899768}{The
  Challenges of Conceptualizing Terrorism}. \emph{Terrorism and
  Political Violence}, 16(4):777-794.
\item
  Jaggar, A. 2005.
  \href{http://onlinelibrary.wiley.com/doi/10.1111/j.1467-9833.2005.00267.x/full}{What
  is Terrorism, Why Is It Wrong, and Could It Ever Be Morally
  Permissible?}. \emph{Journal of Social Philosophy}, 36(2):202-217.
\item
  Hoffmann, B. 2006. \emph{Inside Terrorism}. New York: Columbia
  University Press.
  \href{https://github.com/danilofreire/pols1824w/raw/master/readings/hoffman2006chap1.pdf}{Chapter
  1}.
\item
  Shughart, W. 2006.
  \href{https://link.springer.com/article/10.1007/s11127-006-9043-y}{An
  Analytical History of Terrorism, 1945-2000}. \emph{Public Choice},
  128(1-2):7-39.
\item
  Young, J \& Findley, M. 2011.
  \href{https://doi.org/10.1111/j.1468-2486.2011.01015.x}{Promise and
  Pitfalls of Terrorism Research}. \emph{International Studies Review},
  13(3):411-431.
\end{itemize}

\hypertarget{week-11-discussion-of-final-projects}{%
\subsection{Week 11: Discussion of final
projects}\label{week-11-discussion-of-final-projects}}

\hypertarget{week-12-the-rational-terrorist}{%
\subsection{Week 12: The rational
terrorist}\label{week-12-the-rational-terrorist}}

\begin{itemize}
\tightlist
\item
  Pape, R. 2003.
  \href{https://www.cambridge.org/core/journals/american-political-science-review/article/strategic-logic-of-suicide-terrorism/A6F51C77E3DE644EBD20ADE176973547}{The
  Strategic Logic of Suicide Terrorism}. \emph{American Political
  Science Review}, 97(3):343-361.
\item
  Kydd, A. \& Walter, B. 2006.
  \href{https://www.mitpressjournals.org/doi/abs/10.1162/isec.2006.31.1.49}{The
  Strategies of Terrorism}. \emph{International Security}, 31(1):49-80.
\item
  Horowitz, M. 2010.
  \href{https://www.cambridge.org/core/journals/international-organization/article/nonstate-actors-and-the-diffusion-of-innovations-the-case-of-suicide-terrorism/9D060458C614C2BBD8322ED5D444AA32}{Nonstate
  Actors and the Diffusion of Innovations: The Case of Suicide
  Terrorism}. \emph{International Organization}, 64(1):33-64.
\item
  Gambetta, D. \& Hertog, S 2009.
  \href{https://www.cambridge.org/core/journals/european-journal-of-sociology-archives-europeennes-de-sociologie/article/why-are-there-so-many-engineers-among-islamic-radicals/91ED8BEFDE3793834667750B31575422}{Why
  Are There So Many Engineers Among Islamic Radicals?}. \emph{European
  Journal of Sociology}, 50(2):201-230.
\item
  Horgan, J. 2008.
  \href{http://journals.sagepub.com/doi/abs/10.1177/0002716208317539}{From
  Profiles and Pathways and Roots to Routes: Perspectives from
  Psychology on Radicalization into Terrorism}. \emph{The Annals of the
  American Academy of Political and Social Science}, 618(1):80-94.
\end{itemize}

\hypertarget{week-13-is-terrorism-effective}{%
\subsection{Week 13: Is terrorism
effective?}\label{week-13-is-terrorism-effective}}

\begin{itemize}
\tightlist
\item
  Gould, E. \& Klor, E. 2010.
  \href{https://www.jstor.org/stable/40961012}{Does Terrorism Work?}.
  \emph{Quarterly Journal of Economics}, 125(4):1459--1510.
\item
  Abrahms, M. 2006.
  \href{https://www.mitpressjournals.org/doi/pdf/10.1162/isec.2006.31.2.42}{Why
  Terrorism Does Not Work}. \emph{International Security}, 31(2):42--78.
\item
  Kalyvas, S. 2004.
  \href{https://link.springer.com/article/10.1023/B:JOET.0000012254.69088.41}{The
  Paradox of Terrorism in Civil War}. \emph{The Journal of Ethics},
  8(1):97-138.
\item
  Asal, V. \& Rethemeyer, R. 2008.
  \href{http://journals.sagepub.com/doi/abs/10.1080/07388940802219000}{Dilettantes,
  Ideologues, and the Weak: Terrorists Who Don't Kill}. \emph{Conflict
  Management and Peace Science}, 25(3):244-260.
\item
  Stephan, M. \& Chenoweth, E. 2008.
  \href{https://www.mitpressjournals.org/doi/abs/10.1162/isec.2008.33.1.7}{Why
  Civil Resistance Works: The Strategic Logic of Nonviolent Conflict}.
  \emph{International Security}, 33(1):7-44.
\end{itemize}

\hypertarget{week-14-counterterrorism}{%
\subsection{Week 14: Counterterrorism}\label{week-14-counterterrorism}}

\begin{itemize}
\tightlist
\item
  Lyall, J., Blair, G. \& Imai, K. 2013.
  \href{https://www.cambridge.org/core/journals/american-political-science-review/article/explaining-support-for-combatants-during-wartime-a-survey-experiment-in-afghanistan/B0E55BA87D4EBF66F0BF6135959541A7}{Explaining
  Support for Combatants during Wartime: A Survey Experiment in
  Afghanistan}. \emph{American Political Science Review},
  107(4):679-705.
\item
  Bermann, E., Felter, J. \& Shapiro, J. 2011.
  \href{http://www.jstor.org/stable/10.1086/661983}{Can Hearts and Minds
  be Bought?: The Economics of Counterinsurgency in Iraq}. \emph{Journal
  of Political Economy}, 119(4):766-819.
\item
  Kilcullen, D. 2005.
  \href{https://www.tandfonline.com/doi/abs/10.1080/01402390500300956}{Countering
  Global Insurgency}. \emph{The Journal of Strategic Studies},
  28(4):597-617
\item
  Savun, B. \& Tirone, D. 2017.
  \href{https://doi.org/10.1177/0022002717704952}{Foreign Aid as a
  Counterterrorism Tool: More Liberty, Less Terror?}. \emph{Journal of
  Conflict Resolution}, 62(8):1607-1635.
\item
  O'Donnell, D. 2006.
  \href{https://www.cambridge.org/core/journals/international-review-of-the-red-cross/article/international-treaties-against-terrorism-and-the-use-of-terrorism-during-armed-conflict-and-by-armed-forces/DCEF95D271A0778572F566936A18887F}{International
  Treaties Against Terrorism and the Use of Terrorism During Armed
  Conflict and by Armed Forces}. \emph{International Review of the Red
  Cross}, 88(864):853-880.
\end{itemize}

\hypertarget{week-15-final-project-presentations.}{%
\subsection{Week 15: Final project
presentations.}\label{week-15-final-project-presentations.}}




\end{document}

\makeatletter
\def\@maketitle{%
  \newpage
%  \null
%  \vskip 2em%
%  \begin{center}%
  \let \footnote \thanks
    {\fontsize{18}{20}\selectfont\raggedright  \setlength{\parindent}{0pt} \@title \par}%
}
%\fi
\makeatother
